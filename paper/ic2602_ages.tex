%% \begin{deluxetable}{} command tell LaTeX how many columns
%% there are and how to align them.
\begin{deluxetable}{ccc}
    
%% Keep a portrait orientation

%% Over-ride the default font size
%% Use Default (12pt)

%% Use \tablewidth{?pt} to over-ride the default table width.
%% If you are unhappy with the default look at the end of the
%% *.log file to see what the default was set at before adjusting
%% this value.

%% This is the title of the table.
\caption{Previously reported ages for the open cluster IC~2602.}
\label{tab:ages}

%% This command over-rides LaTeX's natural table count
%% and replaces it with this number.  LaTeX will increment 
%% all other tables after this table based on this number
\tablenum{3}

%% The \tablehead gives provides the column headers.  It
%% is currently set up so that the column labels are on the
%% top line and the units surrounded by ()s are in the 
%% bottom line.  You may add more header information by writing
%% another line between these lines. For each column that requries
%% extra information be sure to include a \colhead{text} command
%% and remember to end any extra lines with \\ and include the 
%% correct number of &s.
\tablehead{\colhead{Method} & \colhead{Age [Myr]} & \colhead{Reference}} 

%% All data must appear between the \startdata and \enddata commands
\startdata
MSTO isochrone & $36.3$ & \citet{mermilliod_comparative_1981} \\
PMS+MSTO isochrone & $30 \pm 5$ & \citet{stauffer_rotational_1997} \\
Isochrone\tablenotemark{a} & $67.6$ & \citet{kharchenko_astrophysical_2005} \\
Isochrone\tablenotemark{b} & $221$ & \citet{Kharchenko_et_al_2013} \\
Isochrone  & $67.6$ & \citet{van_leeuwen_parallaxes_2009} \\
LDB\tablenotemark{c} & $46^{+6}_{-5}$ & \citet{dobbie_ic_2010} \\
MSTO isochrone\tablenotemark{d} & $41-46$ & \citet{david_ages_2015} \\
MSTO isochrone\tablenotemark{e} & $37-43$ & \citet{david_ages_2015} \\
Li selection + isochrone & $43.7^{+4.3}_{-3.9}$ & \citet{bravi_gaia-eso_2018} \\
Isochrone\tablenotemark{f} & $30^{+9}_{-7}$ & \citet{randich_gaiaeso_2018} \\
Isochrone & $35.5^{+0.8}_{-1.6}$ & \citet{bossini_age_2019} \\
\enddata

%% Include any \tablenotetext{key}{text}, \tablerefs{ref list},
%% or \tablecomments{text} between the \enddata and 
%% \end{deluxetable} commands

%% General table comment marker
\tablecomments{
  MSTO $\equiv$ main sequence turn-off.
	LDB $\equiv$ lithium depletion boundary.
}
\tablenotetext{a}{
		Based on location in HR diagram of just two stars.
}
\tablenotetext{b}{
  Notes major age change since \citet{kharchenko_astrophysical_2005}.
}   
\tablenotetext{c}{
  \citet{dobbie_ic_2010} performed a dedicated study of the LDB in IC~2602.  Comparing to early isochronal ages, they write their age is ``consistent with the general trend delineated by the Pleiades, $\alpha$-Per, IC$\,$2391, and NGC$\,$2457, whereby the LDB age is ~120-160 per cent of the estimates derived using more traditional techniques'' such as isochrone-fitting.
}   
\tablenotetext{d}{
  Using \citet{ekstrom_grids_2012} evolutionary models.
}
\tablenotetext{e}{
  Using PARSEC evolutionary models \citep{bressan_parsec_2012}.
}   
\tablenotetext{f}{
  Averaged across PROSECCO, PARSEC, MIST models in $(J, H, K_{\rm s})$ and $(J, H, K_{\rm s}, V)$ planes.
}   
\end{deluxetable}
